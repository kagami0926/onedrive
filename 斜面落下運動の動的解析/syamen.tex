\documentclass{jlreq}
\begin{document}
  \title{斜面降下運動の動的解析}
  \date{2024/6/5}
  \author{加賀美 渉太}
  \maketitle

  \section{目的}
  直線や曲面上を転がる球の運動を解析することを通して運動の動的解析を行う。
  
  \section{原理}
  まずは図のような直線上の斜面を降下する球の運動を考える。小球の質量を$m$、重力加速度を$g$、斜面と水平面がなす角度を$\theta$、初速度$v_0=0$とおいて、時刻$t=0$での小球の位置を原点とし斜面に沿って下方向を正とする$s$軸をとると、小球についての運動方程式は
  \[m\frac{d^2s}{dt^2}=mg\sin \theta \\\]
  両辺sで積分して整理すると
  \[\frac{ds}{dt}=g\sin \theta \\\]
  再度両辺sで積分して整理すると
  \[s=\frac{1}{2}g\sin \theta \\t^2\]

  このように斜面が直線であれば数式で考えることができる。球面でない曲面の場合数式で考えることができない、しかし後に説明する実験を行えば曲面上の斜面を降下する球の運動を数値的に解析することが可能である。
  \section{実験方法}
  図で示されるいろいろな形状をした斜面を降下する小球の運動をデジタルカメラを用いて撮影し、その画像を解析することで小球の速度、加速度の変化をまとめ、小球がどのような運動をしているかを明らかにする。

具体的な実験手順を以下の通りである。
  \begin{enumerate}
    \item 図に示すようにデジタルカメラをレール全体がカメラ視野に入るように設置し固定する。また、レールの下に参照用のスケールを設置する。次にレールを曲げて図のような斜面を作る。
    \item 白色に塗布した小球をレールの始点に起き、デジタルカメラで小球が斜面上を降下する運動を記録する。このとき、デジタルカメラの毎秒を20コマ撮影のカメラを使用する。
    \item 画像をコンピュータに転送し、その画像から座標を読み取る。
    \item 画像を合成し、合成画像を作成する。
  \end{enumerate}
  \section{結果}
  
  \section{考察}
  
\end{document}
